\section{Tutorials 1 (7 III 2019)}
\noindent
\textbf{1.} Consider \textsc{Nim} game.\\
We have $n$ stacks of heights $h_1, h_2 ..., h_n, h \in \{0, 1, 2, ...\}$.\\
$W_E = \{x \in \mathbb{N}^R : \text{Eve has a w.s.}\}$.\\
Let's take $x \in \mathbb{N}^R, x = (h_1, ..., h_n)$.\\
$h_1^b \otimes h_2^b \otimes ... \otimes h_n^b \neq 0 \rightarrow \text{Eve wins from } x$.\footnote{($h^b$ means binary representation)}\\
The proof consists of 3 observations:
\begin{itemize}
	\item Final position (all empty stacks) has xor equal to 0.
	\item From a position s.t. xor $\neq 0$, it is always possible to zero the xor.
	Let xor be equal some $y$. Let $d$ be the position of leftmost (most important)
	bit in binary representation of xor. Thus in some stack, the binary representation
	must have The $d$-th bit activated. We can then deactivate $d-$th bit and set
	appropriate values on all less significant bits (to zero the xor), and the resulting
	stack height will be lower.
	\item From a position s.t. xor $= 0$, all moves lead to xor $\neq 0$. If we take
	a non-zero number of tokens from a stack, it means we alter a non-zero number of
	binary digits in the representation of xor, thus it is no longer 0.
\end{itemize}
Thus, $W_E = \{x \in \mathbb{N}^R : \otimes x \neq \vec{0}\}$.\\

\noindent
\textbf{2.} Represent \textsc{Nim} as a game on graph.\\
The set of positions is the set of all stack configurations. In general, the space of positions is infinite,
but for a fixed game we have finite set of positions.\\
However, this is not enough (in a game on graph, we want $Pos$ to be disjoint set $Pos = Pos_E \cup Pos_A, Pos_E \cap Pos_A = \emptyset$).
Thus we also add information who moves next.\\
Graph: $G = \langle V_E \dot\cup V_A, E \rangle$.\\
Strategy is a function $\sigma : V^*V \rightarrow V$, $\sigma(v_0v_1v_2...v_nv_{n+1}) \rightarrow v$.\\
Let $w = v_1...v_n...$ be a winning position. We say that first player wins if their strategy $L \in V^{w}$.\\
In \textsc{Nim}, the winning condition does not depend on history, so we can collapse the states with the same stacks configuration in the last move
(and of course the same currrent player).\\

\noindent
\textbf{3.} Chocolate game.\\
We have a grid $m \times n$. A player chooses a field and everything on the right and up
is erased. One restriction: you cannot choose position $(1, 1)$. The player who makes last
move wins.
\footnote{
Other definition: players eat chocolate, position $(1, 1)$ is poisoned.
}
Is the game determined? Who has a winning strategy?
\begin{itemize}[leftmargin=*,labelindent=12.5mm]
	\item[$1 \times n$:] Eve wins (obvious)
	\item[$2 \times n$:] Eve wins: she eats the position $(n, 2)$ and always maintains
	the bottom row has one piece more than the top one.
	\item[$\omega \times \omega$:]\footnote{$\omega = \{0, 1, ...\}$} Eve wins: eat the position $(2, 2)$ and then maintain $(1, n); (n, 1)$
	\item[$m \times n$:]\footnote{$m, n \in \mathbb{N}$} Assume the $2$nd player wins. That means that for
	any move made in the first move, there exists such move from a second player, that the second player
	has a winning strategy after the second move.\\
	We can use a strategy stealing argument: First player removes a rectangle larger than $1 \times 1$.
	Then by assumption the second player has a winning strategy. But instead we can start by removing
	only one piece of chocolate. Then second player must make a move such that only a single rectangular
	area is removed (which could be made in one move by the first player). Then first player can copy
	second player's winning strategy. Thus second player does not have a winning strategy.\\
\end{itemize}
Above we have shown that the second player has no w.s.\\
$\lnot \exists_{\pi} \forall_{\sigma} \pi \text{ wins with } \sigma \Leftrightarrow \forall_{\pi} \exists_{\sigma} \sigma \text{ wins with } \pi$.
But this doesn't imply $\exists_{\sigma} \forall_{\pi} \sigma > \pi$ (although there is an implication the other way).\\
We will just show the game is determined, without showing the winning strategy:\\
We can build the graph of positions. All paths from $m \times n$ (starting position)
are finite, the size of graph is finite as well. We can thus infer the winning position by
searching the graph bottom-up (from node $0, 0$ to $m, n$). We have thus shown the game is
determined for a finite size. Since Adam does not have a winning strategy, thus Eve must have
it.\\
For $\omega \times \omega$ we use an argument that after the first move the game graph must have
a finite height.\\

At home: think about chess determinacy, also Armageddon version (black wins if he doesn't lose, also for draw).
