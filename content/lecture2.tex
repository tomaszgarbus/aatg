\section{Lecture 2 (6 III 2019)}
\subsection*{Determinacy}
If we have a \textbf{game of finite duration} with 2 players, we can expand the
game in a tree, where a leaf signifies the end of the game. A leaf maps to one of
three possible situations:
\begin{itemize}
	\item existential player ($\exists$) wins
	\item universal player ($\forall$) wins
	\item draw
\end{itemize}
If we map those situations to values accordingly: $1, -1, 0$, the existential
player aims to maximize (and universal to minimize) the outcome value.\\

\noindent
Let's consider \textbf{infinite} games now. Suppose we have 2 players and draw is
not possible in the game. If the player does not know the winning strategy, it is
possible that they may "loop" in a position with winning strategy but never proceed
with it.\\

\noindent
There \textbf{exist} indeterminate perfect information games.\\

\noindent
Infinite XOR game: $E$ and $A$ alternately play words $w_0, w_1, w_2, ... \in \{0, 1\}^{+}$
which are concatenated to $w_0w_1w_2...$.\\

\noindent
Infinite XOR: any function $f: \{0, 1\}^{|\mathbb{N}|} \rightarrow \{0, 1\}$ such that
if $v, w$ differ by one bit then $f(v) \neq f(w)$.\\
$v \sim w$ iff differ by a finite number of bits.\\
We can choose set $S$ s.t. $\{0, 1\}^{|\mathbb{N}|} \supseteq S$ has $\exists!$ element for each equivalence class (from \textit{Axiom of Choice}).\\
Each equivalence class of $\sim$ is countable, thus there is continuum of equivalence classes.

\noindent
Eve wins iff $f(w_0w_1...) = 0$, Adam otherwise. No player has a winning strategy
in this game.\\
\underline{1. Suppose Adam wins.} In the first play:
\begin{verbatim}
E  0      w_2       w_4
A    w_1       w_3
\end{verbatim}
Then in the next game Eve can steal his strategy:
\begin{verbatim}
E  1 w_1        w_3       w_5
A         w_2        w_4
\end{verbatim}
\underline{2. Suppose Eve wins.} In the first play:
\begin{verbatim}
E  w_0    w_2      w_4
A       0      w_3
\end{verbatim}
Then in the next play:
\begin{verbatim}
E  w_0         w_3
A       1 w_2       w_4
\end{verbatim}

\subsection*{Game on graph}
An \textit{arena} is a directed graph, consisting of:
\begin{itemize}
	\item the set of \textit{positions} $Pos$
	\item the set of \textit{moves} $Moves \subseteq Pos \times Pos$
\end{itemize}
$Pos = Pos_{\exists} \cup Pos_{\forall}$,\\
$Pos_{\exists} \cup Pos_{\forall} \neq \emptyset$.\\
A \textit{play} is a finite or infinite sequence of moves:\\
$q_0 \rightarrow q_1 \rightarrow q_2 \rightarrow ... \rightarrow q_k (\rightarrow ...)$.\\

\noindent
\underline{Game equation}
\begin{figure}[H]
	\centering
	$X = (E \cap \diamond X) \cup (A \cap \square X) = Eve(X)$\\
	$Y = (E \cap \square Y) \cup (A \cap \diamond Y) = Adam(Y)$
\end{figure}
\noindent
$E = Pos_{\exists}$,\\
$A = Pos_{\forall}$,\\
$X, Y \in \mathcal{P}(Pos)$\\
"Modal logic" symbols here:\\
$\diamond Z = \{p : (\exists_q) Moves(p, q) \land q \in Z\}$
\footnote{
	$p \rightarrow q$ also denotes $Moves(p, q)$ below.
	A position $p$, such that $(\forall_p) p\not\rightarrow q$ is called \textit{terminal},
	which we also write $p \not\rightarrow$.
}\\
$\square Z = \{p : (\forall_q) (p \rightarrow q) \Rightarrow q \in Z\}$\\

\noindent
\textbf{Knaster-Tarski Theorem}: $\langle L, \le \rangle$ complete lattice\footnote{
	A \textit{complete lattice} is a partially ordered set $\langle L, \leq \rangle$, such that
	each subset $Z \subseteq L$ has the least upper bound $\bigvee Z$, and the greates lower
	bound $\bigwedge Z$. In particular, $\bigvee \emptyset$ is the least element denoted $\bot$,
	and $\bigwedge \emptyset$ is the greatest element denoted $\top$.
}, $f : L \rightarrow L$ monotonic,
then there exists a least fixed point\\
$\mu x. f(x) = \bigwedge \{d : f(d) \leq d\}$ and a greatest fixed point:\\
$\nu y. f(y) = \bigvee \{d : d \leq f(d) \}$.\\
\textbf{Proof}: We show the proof for the greatest fixed point. Let $a = \bigvee A, A = \{z : z \leq f(z)\}$\\
Because $f$ is monotonic, $z \leq a$ implies $f(z) \leq f(a)$. For $z \in A$, this also means $z \leq f(z) \leq f(a)$.
Hence, $f(a)$ is an upper bound of $A$, which follows $a \leq f(a)$. Using the monotonicity of $f$ once more, we obtain
$f(a) \leq f(f(a))$. Hence $f(a) \in A$, which follows the converse inequality $f(a) \leq a$.\\

\noindent
We consider mappings $Eve$ and $Adam$ defined in the complete lattice $\langle \mathcal{P}, \leq \rangle$.
$Eve(Z)$ is a set of such positions from which Eve can win, and $Adam(Z)$ is a set of such positions from which Adam can win.

\subsection*{Traps and gardens of Eden}
\noindent
A set of positions $Z \subseteq Pos$ is a trap for Adam if $Z \subseteq Eve(Z)$. It means that Adam cannot go out of there.\\
A set of positions $Z \subseteq Pos$ is Garden of Eden for Eve if $Eve(Z) \subseteq Z$. It means that Adam cannot enter those positions.\\

\noindent
The \textit{greatest} trap for Adam is a garden of Eden for Even.\\
The \textit{least} garden of Eden for Eve is a trap for Adam.\\

We use the notation: $\overline{Z} = Pos - Z$.

\subsubsection*{Lemma}
\begin{figure}[H]
	\centering
	$\overline{Eve(X)} = Adam(\overline{X})$
\end{figure}
\textit{Proof.} We have:\\
$\overline{Eve(X)} = \overline{(E \cap \diamond X) \cup (A \cap \square X)}\newline
 = (\overline{E \cap \diamond X}) \cap (\overline{A \cap \square X})\newline
 = (\overline{E} \cup \overline{\diamond X}) \cap (\overline{A} \cup \overline{\square X})\newline
 = (A \cup \square \overline{X}) \cap (E \cup \diamond \overline{X})\newline
 = (A \cap \diamond \overline{X}) \cup (E \cap \diamond \overline{X})
 \cup (A \cap E) \cup (\diamond \overline{X} \cap \square \overline{X}) = Adam(\overline{X})$\\

\noindent
\underline{\textbf{Proposition:}} $Pos$ can be divided to three disjoint sets:
$\mu X. Eve(X)$, $\mu X. Adam(X), (\nu Y. Eve(Y)) \cap (\nu Y. Adam(Y))$\\

\subsection*{Definition: strategy}
A strategy (for Eve) is a set of finite plays s.t.:
\begin{itemize}
	\item if $last(w) \in Pos_{\exists}$ then $\exists! q$ s.t. $last(w) \rightarrow q$ and $wq$ is in $S$
	\item if $last(w) \in Pos_{\forall}$ then $\forall(q) (last(w) \rightarrow q) \Rightarrow wq \in S$
	\item $S$ is closed under initial segments, i.e., if $s_0s_1...s_k \in S$, then $s_0s_1...s_i \in S$,
	for $0 \leq i \leq k$j
\end{itemize}
