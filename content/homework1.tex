\section{Homework1}
\subsection*{5 Multi-reachability games}
\textbf{In a reachability games there is a set of vertices which the first player
wants to reach. In multi-reachability games (MRG) there is a family of
sets of vertices and the first player wins if every set in the family has been
visited at least once.\\
1. Show that MRG game is PSpace-complete.}\\
1. \underline{$\in \textsc{PSpace}$}\\
HMMMMM

\noindent
2. \underline{\textsc{PSpace}-hard}\\
I will show a reduction of QBF problem to an MRG game.\\
An input to the QBF problem is a formula, about which I will make two assumptions:
\begin{itemize}
    \item It is in prenex normal form (i.e. all quantifiers preceed the portion
    containing an unquantified Boolean formula). Moreover, let's assume that the
    existental and universal quantifiers alternate -- if it is not the case in the
    original input formula, we can introduce quantifiers with dummy variables, not
    used anywhere in the formula. For instance, $\exists_{x_1} \exists_{x_2}
    \phi(x_1, x_2) \mapsto \exists_{x_1} \forall_{y_1} \exists_{x_2} \phi(x_1, x_2)$
    ($y_1$ is a "dummy" variable).
    \item The "body" of the formula is in conjunctive normal form.
\end{itemize}
Note that the above assumptions do not reduce the expressive power of input formulas.
Every possible formula can be represented in the described format. QBF problem for
such normalized formulas is still \textsc{PSpace}-complete.\\
Let $\forall_{x_1} \exists_{x_2} \forall_{x_3} ... \exists_{x_n}
(y_{1,1} \lor ... \lor y_{1,k_1}) \land (y_{2,1} \lor ... \lor y_{2,k_2}) \land ... (y_{m,1} \lor ... \lor y_{m,k_m})$
be the input QBF formula, where $y_{...} \in \{x_1, ..., x_n, \lnot x_1, ... \lnot x_n \}$.
The created multi-reachability game is $G = \langle V, E, v_I, S \rangle$, where:
\begin{itemize}
    \item $V$ is the set of vertices. Vertices are indexed by all variables bound by
          quantifiers and their negations, plus there is the initial vertex.
          $V = \underset{1 \leq i \leq n}{\bigcup} \{v_x, v_{\lnot x}\} \cup \{v_I\}$
    \item $E$ is the set of edges.
          $E = \{(v_I, v_{x_1}), (v_I, v_{\lnot x_1})\} \cup
          \underset{2 \leq i \leq n}{\bigcup} \{ (v_{x_{i-1}}, v_{x_i}), (v_{x_{i-1}}, v_{\lnot x_i}),
          (v_{\lnot x_{i-1}}, v_{x_i}), (v_{\lnot x_{i-1}}, v_{\lnot x_i}) \}$
    \item $v_I$ is a starting vertex.
    \item $S$ is the family of sets of vertices that the first player wants to reach.
          It is created directly from the CNF formula, i.e.
          $S = \underset{1 \leq i \leq m}{\bigcup} \{ \{ v_{y_{i,j}}\ |\ 1 \leq j \leq k_i \} \}$
\end{itemize}
The first player is the existential player, their opponent is the universal player (and they start the game).
The QBF formula is satisfiable iff the first player has a winning strategy.