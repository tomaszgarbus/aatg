\section{Homework1}
\subsection*{5 Multi-reachability games}
\textbf{In a reachability games there is a set of vertices which the first player
wants to reach. In multi-reachability games (MRG) there is a family of
sets of vertices and the first player wins if every set in the family has been
visited at least once.\\
1. Show that MRG game is PSpace-complete.}\\
1. \underline{$\in \textsc{PSpace}$}\\
I will use the fact that a normal reachability game is solvable in polynomial time.
I will also use Savitch theorem, in particular its corollary, stating that
$\textsc{PSpace} = \textsc{NPSpace}$.\\
Let $G = \langle V, E, v_I, S \rangle$ be the MRG, where $V$ is the set of vertices,
$E$ is the set of edges, $v_I$ is the starting vertex and $S$ is the family of sets
the first player wants to visit.\\
The first player can nondeterministically choose an ordered sequence of vertices
$v_1, v_2, ..., v_k$ such that $\forall_{T \in S} \exists_{1 \leq i \leq k} (v_i \in T)$.
Then we want to check in polynomial space whether the first player can visit selected
vertices in the specified order, regardless of what the second player is doing.
This can be done by considering around $4k$ normal reachability games. For each $i, 1 \leq i \leq k$,
we create 4 reachability games:
\begin{itemize}
      \item $\langle V', E', (v_{i-1}, 0), \{ (v_i, 0) \} \rangle$
      \item $\langle V', E', (v_{i-1}, 0), \{ (v_i, 1) \} \rangle$
      \item $\langle V', E', (v_{i-1}, 1), \{ (v_i, 0) \} \rangle$
      \item $\langle V', E', (v_{i-1}, 1), \{ (v_i, 1) \} \rangle$
\end{itemize}
where:\\
$V' = V \times \{0,1\}$\\
$E' = \{ ((u,0), (v,1)), ((u,1),(v,0)) \ |\ (u, v) \in E \}$\\
$v_0 = v_I$\\
Player $p$ starts, where $(v_{i-1}, p) = v_I$ (not necessarily the first player from our MRG)\\

\noindent
Such extended set of vertices $V'$ carries, together with the vertex, the information whose move it is.
Having solved the single reachability games described above, we can perform an algorithm in a
dynamic-programming-manner as follows:
\begin{lstlisting}[tabsize=2]
  wins: bool[k][2][2] = given     # wins[i][ps][pe] iff the first
                                  # player has a winning strategy in
                                  # a game <V', E', (v$_{i-1}$, ps), (v$_i$, pe)>

  can_reach[][] = false * [k][2]  # At the end of the algorithm
                                  # can_reach[i][p] = true iff
                                  # first player can force the game
                                  # to reach vertex v$_i$ and player
                                  # number $p$ has the next move.
  
  can_reach[0][0] = true          # First player starts.

  for i := 1 to k:
    for ps := 0 to 1:
      for pe := 0 to 1:
        if can_reach[i-1][ps] and wins[i][ps][pe]:
          can_reach[i][pe] = true
\end{lstlisting}
If either \texttt{can\_reach[k][0]} or \texttt{can\_reach[k][1]} then the first
player has a winning strategy.\\

\noindent
2. \underline{\textsc{PSpace}-hard}\\
I will show a reduction of QBF problem to an MRG game.\\
An input to the QBF problem is a formula, about which I will make two assumptions:
\begin{itemize}
    \item It is in prenex normal form (i.e. all quantifiers preceed the portion
    containing an unquantified Boolean formula). Moreover, let's assume that the
    existental and universal quantifiers alternate -- if it is not the case in the
    original input formula, we can introduce quantifiers with dummy variables, not
    used anywhere in the formula. For instance, $\exists_{x_1} \exists_{x_2}
    \phi(x_1, x_2) \mapsto \exists_{x_1} \forall_{y_1} \exists_{x_2} \phi(x_1, x_2)$
    ($y_1$ is a "dummy" variable).
    \item The "body" of the formula is in conjunctive normal form.
\end{itemize}
Note that the above assumptions do not reduce the expressive power of input formulas.
Every possible formula can be represented in the described format. QBF problem for
such normalized formulas is still \textsc{PSpace}-complete.\\
Let $\forall_{x_1} \exists_{x_2} \forall_{x_3} ... \exists_{x_n}
(y_{1,1} \lor ... \lor y_{1,k_1}) \land (y_{2,1} \lor ... \lor y_{2,k_2}) \land ... (y_{m,1} \lor ... \lor y_{m,k_m})$
be the input QBF formula, where $y_{...} \in \{x_1, ..., x_n, \lnot x_1, ... \lnot x_n \}$.
The created multi-reachability game is $G = \langle V, E, v_I, S \rangle$, where:
\begin{itemize}
    \item $V$ is the set of vertices. Vertices are indexed by all variables bound by
          quantifiers and their negations, plus there is the initial vertex.
          $V = \underset{1 \leq i \leq n}{\bigcup} \{v_{x_i}, v_{\lnot {x_i}}\} \cup \{v_I\}$.
          Note that the vertices are positions and:\\
          $Pos = V$\\
          $v_x \in Pos_{\exists}$ iff $x_i$ is bound by an existential quantifier\\
          $v_x \in Pos_{\forall}$ iff $x_i$ is bound by an universal quantifier
    \item $E$ is the set of edges.
          $E = \{(v_I, v_{x_1}), (v_I, v_{\lnot x_1})\} \cup
          \underset{2 \leq i \leq n}{\bigcup} \{ (v_{x_{i-1}}, v_{x_i}), (v_{x_{i-1}}, v_{\lnot x_i}),
          (v_{\lnot x_{i-1}}, v_{x_i}), (v_{\lnot x_{i-1}}, v_{\lnot x_i}) \}$
    \item $v_I$ is a starting vertex.
    \item $S$ is the family of sets of vertices that the first player wants to reach.
          It is created directly from the CNF formula, i.e.
          $S = \underset{1 \leq i \leq m}{\bigcup} \{ \{ v_{y_{i,j}}\ |\ 1 \leq j \leq k_i \} \}$
\end{itemize}
The first player is the existential player, their opponent is the universal player (and they start the game).
The QBF formula is satisfiable iff the first player has a winning strategy.\\

\noindent
\textbf{2. Show that one-player MRG is NP-complete.}\\
1. \underline{$\in \textsc{NP}$}\\
The algorithm is as follows:\\
\begin{itemize}
      \item[1] We make a nondeterministic selection of one element of every set of vertices to be visited
               and we also choose nondeterministically a permutation of those verices -- the order in which
               we want to visit them. If some vertex was chosen more than once, we only leave one copy of it
               in the chosen order.
      \item[2] It is possible to check in polynomial time whether the chosen vertices can be visited in the
               chosen order. Let $v_1, v_2, ..., v_k$ be the chosen sequence of vertices. For any pair of
               vertices $u, w$, it can be decided in polynomial time whether $w$ is reachable from $u$
               (e.g. a simple depth-first search algorithm). Let $v_0 = v_I$, the starting vertex of the game.
               The player has a winning strategy if $\forall_{1 \leq i \leq k}(v_i \text{ is reachable from } v_{i-1})$.
\end{itemize}

\noindent
2. \underline{\textsc{NP}-hard}\\
I will show the reduction of 3SAT problem (satisfiability of CNF formula where each clause has at most 3 literals) to
one-player MRG. The resulting game will be very similar as in previous subproblem.\\
Let $(y_{1,1} \lor y_{1,2} \lor y_{1,3}) \land ... \land (y_{m,1} \lor y_{m,2} \lor y_{m,3})$ be the input formula,
where $\underset{\substack{1 \leq i \leq m\\1 \leq j \leq 3}}{\forall} y_{i,j} \in \underset{1 \leq k \leq n}{\bigcup} \{ x_k, \lnot x_k \}$.\\
We create a one-player MRG as follows:\\
$G = \langle V, E, v_I, S \rangle$, where:
\begin{itemize}
      \item $V = \underset{1 \leq i \leq n}{\bigcup} \{v_x, v_{\lnot x}\} \cup \{v_I\}$ is the set of vertices.
            Each vertex (except for the initial one) corresponds to some variable or its negation. Visiting a
            vertex is synonymous to assigning a value to the variable.
      \item $E$ is the set of edges.
            $E = \{(v_I, v_{x_1}), (v_I, v_{\lnot x_1})\} \cup
            \underset{2 \leq i \leq n}{\bigcup} \{ (v_{x_{i-1}}, v_{x_i}), (v_{x_{i-1}}, v_{\lnot x_i}),
            (v_{\lnot x_{i-1}}, v_{x_i}), (v_{\lnot x_{i-1}}, v_{\lnot x_i}) \}$.
      \item $v_I$ is the initial vertex.
      \item $S$ is the family of sets of vertices that the player wants to reach.
            It is constructed directly from the CNF formula, i.e.
            $S = \underset{1 \leq i \leq m}{\bigcup} \{ \{ v_{y_{i,j}}\ |\ 1 \leq j \leq 3 \} \}$
\end{itemize}
The formula is satisfiable iff the player can find a path in the graph, starting from $v_I$ and visiting
at least one vertex from each set in $S$. Moreover, the indices of visited vertices form a satisfying valuation
of the variables occuring in the input formula.\\

\noindent
\textbf{3. Show that if the sets are singletons then MRG is P-complete.}\\

\noindent
1. \underline{$\in \textsc{P}$}\\
I will again use the fact that a single reachability game is decidable in polynomial time.
Let $G = \langle V, E, v_I, S = \{ \{v_1\}, \{v_2\}, ..., \{v_k\} \}\rangle$. Just like in the first
subproblem, let's create some single reachability games:\\
For each $i, 1 \leq i \leq k$, we create 4 reachability games:
\begin{itemize}
      \item $\langle V', E', (v_{i-1}, 0), \{ (v_i, 0) \} \rangle$
      \item $\langle V', E', (v_{i-1}, 0), \{ (v_i, 1) \} \rangle$
      \item $\langle V', E', (v_{i-1}, 1), \{ (v_i, 0) \} \rangle$
      \item $\langle V', E', (v_{i-1}, 1), \{ (v_i, 1) \} \rangle$
\end{itemize}
where:\\
$V' = V \times \{0,1\}$\\
$E' = \{ ((u,0), (v,1)), ((u,1),(v,0)) \ |\ (u, v) \in E \}$\\
$v_0 = v_I$\\
Player $p$ starts, where $(v_{i-1}, p) = v_I$ (not necessarily the first player from our MRG)\\

\noindent
If (and only if) the first player wins a game $\langle V', E', (u, p), \{ (v, q) \} \rangle$, it means that,
starting from node $u$ in the original game, if it's turn of player $p$, the first player has a strategy
for the game to always reach the node $v$ in a way that player $q$ moves next.\\
Having solved the above games, we can create a relation $R$ such that $((u, p), (v, q)) \in R$ iff
first player has a winning strategy in game $\langle V', E', (u, p), \{ (v, q) \} \rangle$.
Note that this relation is transitive. We turn this relation into a graph, such that the nodes
are pairs $(v_i, p)$ ($0 \leq k, p \in \{0, 1\}$) and the directed edge between two nodes iff they
are in relation $R$. Then we turn this graph into a DAG of strongly connected components.
Since the relation is transitive, every strongly connected component will also be a clique, so finding
a Hamiltonian cycle is trivial inside each component. There exists a Hamiltonian path in the graph if and only if
the DAG of strongly connected components is a single path. The first player has the winning strategy in the
original MRG if and only if there exists a Hamiltonian path in the constructed DAG.
Every step of this method is polynomial in time.\\

\noindent
2. \underline{P-hard}\\
