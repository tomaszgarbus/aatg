\section{Tutorials 3 (14 III 2019)}
Ordinal numbers:\\
$0, 1, ..., \omega, \omega+1, ..., \omega+n, ... 2 \omega, ..., n \omega, \omega^2, ..., \omega^5, ..., \omega^\omega, ...$\\
$\alpha$\ \ \ $\alpha + 1$\ $|$\ $\alpha \cup \{\alpha\}$  $0 = \{\}$\\
Set of ordinals is well founded (no infinite descending sequence).\\
$\beta$\ \ \ $\bigcup\limits_{\alpha < \beta} \alpha$\\

\noindent
\textbf{1.} \textsc{Nim} on ordinals. $G = (h_1, .., h_n), h_i < \omega^\omega$.
\begin{itemize}
	\item[1)] Is $G$ determined (for which starting positions)?
	\item[2)] If so, who has a winning strategy?
\end{itemize}
1. \underline{Yes.} A graph of states is a DAG and can be divided into disjoint states of winning positions for both players.\\
2.
\begin{itemize}
	\item[a)] $h < \omega^\omega$\\
		$h = a_0 + a_1\omega + a_2\omega^2 + ... + a_n\omega^n\ \ (a_i \in \mathbb{N})$ (it is easy to show that this fits under $\omega^\omega$)\\
		In finite case ($a_i = 0$ if $i > 0$) we aim to zero the xor of all stacks. We will try to apply this strategy to ordinals. Let's
		define xor on ordinals.\\
		$\alpha, \beta$ -- ordinals\\
		$\alpha \oplus \beta = (\alpha_0 \oplus \beta_0) + (\alpha_1 \oplus \beta_1)\omega + ... + (\alpha_n \oplus \beta_n)\omega^n$\\
		\underline{Statement} In game $G = (h_1, .., h_n)$ Eve has a winning strategy iff $\underset{i > 0}{\oplus} h_i \neq \emptyset$.
		We need to prove that:
		\begin{itemize}
			\item[1\textdegree] If Eve plays from $\oplus \neq 0$ then she can always maintain $\oplus = 0$.
			\item[2\textdegree] $\oplus = 0 \Rightarrow$ Every Adam's move makes it $\neq 0$.
			\item[3\textdegree] Ends after a finite number of steps in necessarily Adam's position.
		\end{itemize}
\end{itemize}

\noindent
\textbf{2.} Infinite XOR is an undetermined game such that its graph of positions has inifinite branching. Try to find a game such that
is undetermined but has finite branching.\\
We modify the game so that player can only place a single letter from the alphabet $\Gamma = \{0, 1, 2, 3, \#\}$ and we have an interpretation function
$f : \{0, 1, 2, 3, \#\}^\omega \rightarrow \{0, 1\}^\omega$. If a player does not place a hash (end of a word) in a finite time, they lose. Once a player
places a hash, its the other player's turn.\\

\noindent
\textbf{Gale-Stewart games}: $\langle \Gamma, W \subseteq \Gamma^\omega \rangle$, game of perfect information.\\

\noindent
\textbf{Zermelo}: Let $G = \langle V, \rightarrow \rangle$ be a graph game. Then there exists a partition $W_E \dot\cup W_A \dot\cup W_N = V$
s.t. player $P \in \{E, A\}$ has a \underline{positional} winning strategy in position $v \in W_P$.\\

\noindent
\textbf{Reachability game}: We select a node in graph, Eve wants to reach that node, Adam does not want to ever reach that node. If the play is infinite and looped
without reaching the selected node, Adam wins.
